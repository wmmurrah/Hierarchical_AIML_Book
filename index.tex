% Options for packages loaded elsewhere
\PassOptionsToPackage{unicode}{hyperref}
\PassOptionsToPackage{hyphens}{url}
\PassOptionsToPackage{dvipsnames,svgnames,x11names}{xcolor}
%
\documentclass[
  letterpaper,
  DIV=11,
  numbers=noendperiod]{scrreprt}

\usepackage{amsmath,amssymb}
\usepackage{iftex}
\ifPDFTeX
  \usepackage[T1]{fontenc}
  \usepackage[utf8]{inputenc}
  \usepackage{textcomp} % provide euro and other symbols
\else % if luatex or xetex
  \usepackage{unicode-math}
  \defaultfontfeatures{Scale=MatchLowercase}
  \defaultfontfeatures[\rmfamily]{Ligatures=TeX,Scale=1}
\fi
\usepackage{lmodern}
\ifPDFTeX\else  
    % xetex/luatex font selection
\fi
% Use upquote if available, for straight quotes in verbatim environments
\IfFileExists{upquote.sty}{\usepackage{upquote}}{}
\IfFileExists{microtype.sty}{% use microtype if available
  \usepackage[]{microtype}
  \UseMicrotypeSet[protrusion]{basicmath} % disable protrusion for tt fonts
}{}
\makeatletter
\@ifundefined{KOMAClassName}{% if non-KOMA class
  \IfFileExists{parskip.sty}{%
    \usepackage{parskip}
  }{% else
    \setlength{\parindent}{0pt}
    \setlength{\parskip}{6pt plus 2pt minus 1pt}}
}{% if KOMA class
  \KOMAoptions{parskip=half}}
\makeatother
\usepackage{xcolor}
\setlength{\emergencystretch}{3em} % prevent overfull lines
\setcounter{secnumdepth}{5}
% Make \paragraph and \subparagraph free-standing
\makeatletter
\ifx\paragraph\undefined\else
  \let\oldparagraph\paragraph
  \renewcommand{\paragraph}{
    \@ifstar
      \xxxParagraphStar
      \xxxParagraphNoStar
  }
  \newcommand{\xxxParagraphStar}[1]{\oldparagraph*{#1}\mbox{}}
  \newcommand{\xxxParagraphNoStar}[1]{\oldparagraph{#1}\mbox{}}
\fi
\ifx\subparagraph\undefined\else
  \let\oldsubparagraph\subparagraph
  \renewcommand{\subparagraph}{
    \@ifstar
      \xxxSubParagraphStar
      \xxxSubParagraphNoStar
  }
  \newcommand{\xxxSubParagraphStar}[1]{\oldsubparagraph*{#1}\mbox{}}
  \newcommand{\xxxSubParagraphNoStar}[1]{\oldsubparagraph{#1}\mbox{}}
\fi
\makeatother

\usepackage{color}
\usepackage{fancyvrb}
\newcommand{\VerbBar}{|}
\newcommand{\VERB}{\Verb[commandchars=\\\{\}]}
\DefineVerbatimEnvironment{Highlighting}{Verbatim}{commandchars=\\\{\}}
% Add ',fontsize=\small' for more characters per line
\usepackage{framed}
\definecolor{shadecolor}{RGB}{241,243,245}
\newenvironment{Shaded}{\begin{snugshade}}{\end{snugshade}}
\newcommand{\AlertTok}[1]{\textcolor[rgb]{0.68,0.00,0.00}{#1}}
\newcommand{\AnnotationTok}[1]{\textcolor[rgb]{0.37,0.37,0.37}{#1}}
\newcommand{\AttributeTok}[1]{\textcolor[rgb]{0.40,0.45,0.13}{#1}}
\newcommand{\BaseNTok}[1]{\textcolor[rgb]{0.68,0.00,0.00}{#1}}
\newcommand{\BuiltInTok}[1]{\textcolor[rgb]{0.00,0.23,0.31}{#1}}
\newcommand{\CharTok}[1]{\textcolor[rgb]{0.13,0.47,0.30}{#1}}
\newcommand{\CommentTok}[1]{\textcolor[rgb]{0.37,0.37,0.37}{#1}}
\newcommand{\CommentVarTok}[1]{\textcolor[rgb]{0.37,0.37,0.37}{\textit{#1}}}
\newcommand{\ConstantTok}[1]{\textcolor[rgb]{0.56,0.35,0.01}{#1}}
\newcommand{\ControlFlowTok}[1]{\textcolor[rgb]{0.00,0.23,0.31}{\textbf{#1}}}
\newcommand{\DataTypeTok}[1]{\textcolor[rgb]{0.68,0.00,0.00}{#1}}
\newcommand{\DecValTok}[1]{\textcolor[rgb]{0.68,0.00,0.00}{#1}}
\newcommand{\DocumentationTok}[1]{\textcolor[rgb]{0.37,0.37,0.37}{\textit{#1}}}
\newcommand{\ErrorTok}[1]{\textcolor[rgb]{0.68,0.00,0.00}{#1}}
\newcommand{\ExtensionTok}[1]{\textcolor[rgb]{0.00,0.23,0.31}{#1}}
\newcommand{\FloatTok}[1]{\textcolor[rgb]{0.68,0.00,0.00}{#1}}
\newcommand{\FunctionTok}[1]{\textcolor[rgb]{0.28,0.35,0.67}{#1}}
\newcommand{\ImportTok}[1]{\textcolor[rgb]{0.00,0.46,0.62}{#1}}
\newcommand{\InformationTok}[1]{\textcolor[rgb]{0.37,0.37,0.37}{#1}}
\newcommand{\KeywordTok}[1]{\textcolor[rgb]{0.00,0.23,0.31}{\textbf{#1}}}
\newcommand{\NormalTok}[1]{\textcolor[rgb]{0.00,0.23,0.31}{#1}}
\newcommand{\OperatorTok}[1]{\textcolor[rgb]{0.37,0.37,0.37}{#1}}
\newcommand{\OtherTok}[1]{\textcolor[rgb]{0.00,0.23,0.31}{#1}}
\newcommand{\PreprocessorTok}[1]{\textcolor[rgb]{0.68,0.00,0.00}{#1}}
\newcommand{\RegionMarkerTok}[1]{\textcolor[rgb]{0.00,0.23,0.31}{#1}}
\newcommand{\SpecialCharTok}[1]{\textcolor[rgb]{0.37,0.37,0.37}{#1}}
\newcommand{\SpecialStringTok}[1]{\textcolor[rgb]{0.13,0.47,0.30}{#1}}
\newcommand{\StringTok}[1]{\textcolor[rgb]{0.13,0.47,0.30}{#1}}
\newcommand{\VariableTok}[1]{\textcolor[rgb]{0.07,0.07,0.07}{#1}}
\newcommand{\VerbatimStringTok}[1]{\textcolor[rgb]{0.13,0.47,0.30}{#1}}
\newcommand{\WarningTok}[1]{\textcolor[rgb]{0.37,0.37,0.37}{\textit{#1}}}

\providecommand{\tightlist}{%
  \setlength{\itemsep}{0pt}\setlength{\parskip}{0pt}}\usepackage{longtable,booktabs,array}
\usepackage{calc} % for calculating minipage widths
% Correct order of tables after \paragraph or \subparagraph
\usepackage{etoolbox}
\makeatletter
\patchcmd\longtable{\par}{\if@noskipsec\mbox{}\fi\par}{}{}
\makeatother
% Allow footnotes in longtable head/foot
\IfFileExists{footnotehyper.sty}{\usepackage{footnotehyper}}{\usepackage{footnote}}
\makesavenoteenv{longtable}
\usepackage{graphicx}
\makeatletter
\newsavebox\pandoc@box
\newcommand*\pandocbounded[1]{% scales image to fit in text height/width
  \sbox\pandoc@box{#1}%
  \Gscale@div\@tempa{\textheight}{\dimexpr\ht\pandoc@box+\dp\pandoc@box\relax}%
  \Gscale@div\@tempb{\linewidth}{\wd\pandoc@box}%
  \ifdim\@tempb\p@<\@tempa\p@\let\@tempa\@tempb\fi% select the smaller of both
  \ifdim\@tempa\p@<\p@\scalebox{\@tempa}{\usebox\pandoc@box}%
  \else\usebox{\pandoc@box}%
  \fi%
}
% Set default figure placement to htbp
\def\fps@figure{htbp}
\makeatother
% definitions for citeproc citations
\NewDocumentCommand\citeproctext{}{}
\NewDocumentCommand\citeproc{mm}{%
  \begingroup\def\citeproctext{#2}\cite{#1}\endgroup}
\makeatletter
 % allow citations to break across lines
 \let\@cite@ofmt\@firstofone
 % avoid brackets around text for \cite:
 \def\@biblabel#1{}
 \def\@cite#1#2{{#1\if@tempswa , #2\fi}}
\makeatother
\newlength{\cslhangindent}
\setlength{\cslhangindent}{1.5em}
\newlength{\csllabelwidth}
\setlength{\csllabelwidth}{3em}
\newenvironment{CSLReferences}[2] % #1 hanging-indent, #2 entry-spacing
 {\begin{list}{}{%
  \setlength{\itemindent}{0pt}
  \setlength{\leftmargin}{0pt}
  \setlength{\parsep}{0pt}
  % turn on hanging indent if param 1 is 1
  \ifodd #1
   \setlength{\leftmargin}{\cslhangindent}
   \setlength{\itemindent}{-1\cslhangindent}
  \fi
  % set entry spacing
  \setlength{\itemsep}{#2\baselineskip}}}
 {\end{list}}
\usepackage{calc}
\newcommand{\CSLBlock}[1]{\hfill\break\parbox[t]{\linewidth}{\strut\ignorespaces#1\strut}}
\newcommand{\CSLLeftMargin}[1]{\parbox[t]{\csllabelwidth}{\strut#1\strut}}
\newcommand{\CSLRightInline}[1]{\parbox[t]{\linewidth - \csllabelwidth}{\strut#1\strut}}
\newcommand{\CSLIndent}[1]{\hspace{\cslhangindent}#1}

\KOMAoption{captions}{tableheading}
\makeatletter
\@ifpackageloaded{bookmark}{}{\usepackage{bookmark}}
\makeatother
\makeatletter
\@ifpackageloaded{caption}{}{\usepackage{caption}}
\AtBeginDocument{%
\ifdefined\contentsname
  \renewcommand*\contentsname{Table of contents}
\else
  \newcommand\contentsname{Table of contents}
\fi
\ifdefined\listfigurename
  \renewcommand*\listfigurename{List of Figures}
\else
  \newcommand\listfigurename{List of Figures}
\fi
\ifdefined\listtablename
  \renewcommand*\listtablename{List of Tables}
\else
  \newcommand\listtablename{List of Tables}
\fi
\ifdefined\figurename
  \renewcommand*\figurename{Figure}
\else
  \newcommand\figurename{Figure}
\fi
\ifdefined\tablename
  \renewcommand*\tablename{Table}
\else
  \newcommand\tablename{Table}
\fi
}
\@ifpackageloaded{float}{}{\usepackage{float}}
\floatstyle{ruled}
\@ifundefined{c@chapter}{\newfloat{codelisting}{h}{lop}}{\newfloat{codelisting}{h}{lop}[chapter]}
\floatname{codelisting}{Listing}
\newcommand*\listoflistings{\listof{codelisting}{List of Listings}}
\makeatother
\makeatletter
\makeatother
\makeatletter
\@ifpackageloaded{caption}{}{\usepackage{caption}}
\@ifpackageloaded{subcaption}{}{\usepackage{subcaption}}
\makeatother

\usepackage{bookmark}

\IfFileExists{xurl.sty}{\usepackage{xurl}}{} % add URL line breaks if available
\urlstyle{same} % disable monospaced font for URLs
\hypersetup{
  pdftitle={Hierarchical\_AIML\_Book},
  pdfauthor={William M. Murrah},
  colorlinks=true,
  linkcolor={blue},
  filecolor={Maroon},
  citecolor={Blue},
  urlcolor={Blue},
  pdfcreator={LaTeX via pandoc}}


\title{Hierarchical\_AIML\_Book}
\author{William M. Murrah}
\date{2025-02-23}

\begin{document}
\maketitle

\renewcommand*\contentsname{Table of contents}
{
\hypersetup{linkcolor=}
\setcounter{tocdepth}{2}
\tableofcontents
}

\bookmarksetup{startatroot}

\chapter*{Preface}\label{preface}
\addcontentsline{toc}{chapter}{Preface}

\markboth{Preface}{Preface}

This is a Quarto book.

To learn more about Quarto books visit
\url{https://quarto.org/docs/books}.

\bookmarksetup{startatroot}

\chapter{Introduction}\label{introduction}

\bookmarksetup{startatroot}

\chapter{Summary}\label{summary}

In summary, this book has no content whatsoever.

\bookmarksetup{startatroot}

\chapter{Logistic, Gompertz, and Richards Growth Models: Understanding
Theoretical
Parameters}\label{logistic-gompertz-and-richards-growth-models-understanding-theoretical-parameters}

\section{Overview}\label{overview}

This chapter provides a hands-on approach to modeling growth curves
using logistic, Gompertz, and Richards functions, focusing on the
interpretability of theoretical parameters such as the asymptote,
inflection point, growth rate, and shape parameter. The chapter is
structured in parallel for R (using the brms package) and Python (using
PyMC3 and TensorFlow Probability) to facilitate learning across both
programming environments.

\section{Objectives}\label{objectives}

\begin{enumerate}
\def\labelenumi{\arabic{enumi}.}
\tightlist
\item
  Understand the theoretical background and functional form equations
  for logistic, Gompertz, and Richards curves.
\item
  Learn how to specify, fit, and interpret nonlinear growth models in R
  and Python.
\item
  Explore differences in growth parameters by demographic groups using
  hierarchical modeling.
\item
  Compare models using Bayesian model comparison techniques.
\item
  Extend analyses to Gaussian Processes and Neural Networks while
  retaining interpretability.
\end{enumerate}

\section{Structure of the Chapter}\label{structure-of-the-chapter}

\begin{enumerate}
\def\labelenumi{\arabic{enumi}.}
\tightlist
\item
  \textbf{Introduction and Theoretical Background}
\item
  \textbf{Data Preparation and Exploration}
\item
  \textbf{Logistic Growth Modeling}
\item
  \textbf{Gompertz Growth Modeling}
\item
  \textbf{Richards Curve Modeling}
\item
  \textbf{Comparison and Interpretation}
\item
  \textbf{Extensions: Nonparametric and Neural Networks}
\item
  \textbf{Conclusions and Insights}
\item
  \textbf{References and Further Reading}
\end{enumerate}

\section{Requirements}\label{requirements}

\subsection{R Packages}\label{r-packages}

\begin{itemize}
\tightlist
\item
  \texttt{tidyverse}, \texttt{data.table}, \texttt{ggplot2},
  \texttt{EdSurvey}, \texttt{mice}
\item
  \texttt{brms}, \texttt{loo}, \texttt{bayesplot}, \texttt{cmdstanr}
\end{itemize}

\subsection{Python Packages}\label{python-packages}

\begin{itemize}
\tightlist
\item
  \texttt{pandas}, \texttt{numpy}, \texttt{matplotlib}, \texttt{seaborn}
\item
  \texttt{pymc3}, \texttt{arviz}, \texttt{tensorflow-probability}
\item
  Optional for Advanced Modeling: \texttt{pyro-ppl}, \texttt{gpytorch}
\end{itemize}

\begin{center}\rule{0.5\linewidth}{0.5pt}\end{center}

\section{Part 1: Introduction and Theoretical
Background}\label{part-1-introduction-and-theoretical-background}

\subsection{Motivating Example: Educational Disparities and the ECLS-K
Dataset}\label{motivating-example-educational-disparities-and-the-ecls-k-dataset}

Educational disparities in academic achievement are a persistent issue
with far-reaching implications for economic and societal outcomes.
Students from lower socioeconomic backgrounds and marginalized racial
groups often face systemic disadvantages, resulting in achievement gaps
that influence lifelong opportunities, including access to higher
education, employment prospects, and income potential. Understanding and
intervening in educational systems to alleviate these disparities
requires a nuanced examination of how academic growth unfolds across
different demographic groups.

The Early Childhood Longitudinal Study Kindergarten Class (ECLS-K)
dataset provides a rich source of longitudinal data to investigate these
issues. By modeling growth trajectories in academic achievement,
researchers can identify critical developmental periods and examine how
factors such as socioeconomic status (SES) and race influence growth
parameters. In this context, SES and race serve as proxies for access to
resources and exposure to societal factors, including school quality,
neighborhood environment, and family support systems.

\subsection{Research Questions}\label{research-questions}

This chapter is motivated by the following research questions:\\
- How do academic achievement trajectories differ by SES and race?\\
- Are there gender differences in academic achievement trajectories?\\
- How do academic achievement trajectories differ by SES and race?\\
- When do the most rapid changes in academic achievement occur for
different demographic groups?\\
- Do students from different SES or racial backgrounds reach different
levels of achievement asymptotically?

\subsection{Why Use Functional Form
Models?}\label{why-use-functional-form-models}

Functional form models such as logistic, Gompertz, and Richards curves
provide a powerful framework for examining these research questions.
Unlike more flexible machine learning models, functional form models are
interpretable, with parameters directly linked to theoretical
constructs:\\
- \textbf{Inflection Point}: The timing of the most rapid change,
indicating critical periods for intervention.\\
- \textbf{Asymptote}: The maximum achievement level, highlighting
disparities in long-term educational outcomes.\\
- \textbf{Growth Rate}: The speed of academic progress, providing
insights into the effectiveness of educational environments.\\
- \textbf{Shape Parameter (Richards Curve)}: Captures asymmetry in
growth, potentially revealing differences in early vs.~late acceleration
of achievement.

\subsection{Theoretical and Practical
Implications}\label{theoretical-and-practical-implications}

Using functional form models not only enhances interpretability but also
supports theoretical development by linking empirical findings to
educational and cognitive development theories. For example, differences
in inflection points may indicate developmental milestones influenced by
resource access or educational quality, while variations in asymptotes
may reflect systemic inequalities. Additionally, the parametric nature
of these models facilitates cross-study comparisons, enabling cumulative
knowledge building.

By focusing on functional form models, this chapter aims to provide
researchers with tools that offer both flexibility in modeling nonlinear
growth and interpretability essential for informing educational policy
and practice.

\subsection{Overview of Nonlinear Modeling
Methods}\label{overview-of-nonlinear-modeling-methods}

Nonlinear modeling is a powerful approach to understanding complex
growth patterns and developmental trajectories. Traditional linear
models are often inadequate for capturing the dynamic, non-constant
rates of change observed in natural phenomena, such as cognitive
development and academic achievement. To address this, various nonlinear
modeling techniques have been developed, ranging from spline models and
polynomial regression to more flexible machine learning approaches, such
as neural networks and Gaussian processes.

\textbf{Spline Models and Polynomial Regression:} Spline models and
polynomial regressions provide flexible tools for modeling nonlinearity
by segmenting data into pieces or by fitting higher-order polynomials.
However, these approaches often suffer from overfitting and lack clear
interpretability of parameters related to growth processes.

\textbf{Machine Learning Approaches:} Modern machine learning methods,
such as Random Forests, Gradient Boosting Machines, and Deep Neural
Networks, can capture complex nonlinear relationships and interactions.
Although these models offer high predictive accuracy, they are often
criticized for their lack of interpretability and theoretical grounding,
particularly in educational and cognitive development research.

\subsection{Why Focus on Functional Form
Models?}\label{why-focus-on-functional-form-models}

Unlike purely data-driven approaches, functional form models like the
logistic, Gompertz, and Richards curves offer a balance between
flexibility and interpretability. These models are grounded in
theoretical equations that describe growth processes with interpretable
parameters, such as the asymptote (maximum level of achievement), the
inflection point (timing of most rapid growth), and the growth rate
(speed of development). Importantly, the parameters of these models are
directly related to theoretical constructs, making them highly relevant
for hypothesis testing and theory development.

\subsection{Interoperability and Implications for Theory
Development}\label{interoperability-and-implications-for-theory-development}

A key advantage of using functional form models is the interoperability
of their parameters across different studies and populations. For
example, comparing the inflection point across demographic groups
provides insights into differences in the timing of developmental
milestones. This interpretability fosters theoretical advancements,
enabling researchers to link empirical findings to cognitive and
educational theories. Additionally, the use of parametric functional
forms facilitates replication and comparison across studies,
contributing to a cumulative body of knowledge.

By focusing on logistic, Gompertz, and Richards curves, this chapter
aims to equip researchers with tools that offer both flexibility in
modeling nonlinear growth and interpretability crucial for theory
development.

\subsection{The Logistic Curve}\label{the-logistic-curve}

\begin{itemize}
\tightlist
\item
  Equation:
\end{itemize}

\[
y(t) = \frac{\alpha}{1 + \exp(-\beta (t - \gamma))}
\]

\begin{itemize}
\tightlist
\item
  Parameters:

  \begin{itemize}
  \tightlist
  \item
    \(\alpha\): Asymptote (Maximum achievement level)\\
  \item
    \(\beta\): Growth rate
  \item
    \(\gamma\): Inflection point (Timing of most rapid development)
  \end{itemize}
\end{itemize}

\subsection{The Gompertz Curve}\label{the-gompertz-curve}

\begin{itemize}
\tightlist
\item
  Equation:
\end{itemize}

\[
y(t) = \alpha \exp(-\exp(-\beta (t - \gamma)))
\]

\begin{itemize}
\tightlist
\item
  Parameters:

  \begin{itemize}
  \tightlist
  \item
    \(\alpha\): Asymptote
  \item
    \(\beta\): Growth rate
  \item
    \(\gamma\): Inflection point
  \end{itemize}
\end{itemize}

\subsection{The Richards Curve}\label{the-richards-curve}

\begin{itemize}
\item
  Equation: \[
  y(t) = \frac{\alpha}{(1 + \delta \exp(-\beta (t - \gamma)))^{\frac{1}{\delta}}}
  \]
\item
  Parameters:

  \begin{itemize}
  \tightlist
  \item
    \(\alpha\): Asymptote
  \item
    \(\beta\): Growth rate
  \item
    \(\gamma\): Inflection point
  \item
    \(\delta\): Shape parameter (Asymmetry)
  \end{itemize}
\end{itemize}

\begin{center}\rule{0.5\linewidth}{0.5pt}\end{center}

\section{Part 2: Data Preparation and
Exploration}\label{part-2-data-preparation-and-exploration}

\subsection{Data Source}\label{data-source}

\begin{itemize}
\tightlist
\item
  Using the 2011 ECLS-K dataset for educational achievement.
\item
  Data will be prepared and explored using:

  \begin{itemize}
  \tightlist
  \item
    \textbf{R}: \texttt{tidyverse}, \texttt{EdSurvey}
  \item
    \textbf{Python}: \texttt{pandas}, \texttt{seaborn}
  \end{itemize}
\end{itemize}

\subsection{Preprocessing Steps}\label{preprocessing-steps}

\begin{itemize}
\tightlist
\item
  Loading data and initial exploration.
\item
  Handling missing data using multiple imputation.
\item
  Visualizing growth trajectories by demographic groups.
\end{itemize}

\section{Next Steps}\label{next-steps}

The next section will cover detailed coding examples for Logistic Growth
Modeling in both R and Python. Stay tuned!

\bookmarksetup{startatroot}

\chapter{Logistic Growth Modeling: Understanding Theoretical
Parameters}\label{logistic-growth-modeling-understanding-theoretical-parameters}

\section{Overview}\label{overview-1}

This tutorial focuses on modeling growth curves using the Logistic
function, emphasizing the interpretability of theoretical parameters
such as the asymptote, inflection point, and growth rate. It builds on
the foundational knowledge from the introductory tutorial on nonlinear
growth models, with a hands-on approach for implementation in both R
(using the brms package) and Python (using PyMC3).

\section{Objectives}\label{objectives-1}

\begin{enumerate}
\def\labelenumi{\arabic{enumi}.}
\tightlist
\item
  Understand the functional form and theoretical parameters of the
  Logistic curve.
\item
  Specify, fit, and interpret Logistic growth models in R and Python.
\item
  Investigate differences in growth parameters by demographic groups,
  including SES, race, and gender.
\item
  Compare models using Bayesian model comparison techniques and evaluate
  model fit with posterior predictive checks.
\end{enumerate}

\section{Structure of the Tutorial}\label{structure-of-the-tutorial}

\begin{enumerate}
\def\labelenumi{\arabic{enumi}.}
\tightlist
\item
  \textbf{Introduction to the Logistic Curve}
\item
  \textbf{Model Specification}
\item
  \textbf{Implementation in R using brms}
\item
  \textbf{Implementation in Python using PyMC3}
\item
  \textbf{Parameter Interpretation and Visualization}
\item
  \textbf{Model Comparison and Diagnostics}
\item
  \textbf{Extensions and Advanced Topics}
\item
  \textbf{Conclusions and Insights}
\item
  \textbf{References and Further Reading}
\end{enumerate}

\section{Requirements}\label{requirements-1}

\subsection{R Packages}\label{r-packages-1}

\begin{itemize}
\tightlist
\item
  \texttt{tidyverse}, \texttt{ggplot2}
\item
  \texttt{brms}, \texttt{loo}, \texttt{bayesplot}, \texttt{cmdstanr}
\end{itemize}

\subsection{Python Packages}\label{python-packages-1}

\begin{itemize}
\tightlist
\item
  \texttt{pandas}, \texttt{numpy}, \texttt{matplotlib}, \texttt{seaborn}
\item
  \texttt{pymc3}, \texttt{arviz}
\end{itemize}

\begin{center}\rule{0.5\linewidth}{0.5pt}\end{center}

\section{Part 1: Introduction to the Logistic
Curve}\label{part-1-introduction-to-the-logistic-curve}

\subsection{Motivation and Context}\label{motivation-and-context}

The Logistic growth curve is widely used in modeling growth processes,
particularly in educational achievement and cognitive development, where
growth accelerates rapidly before slowing as it approaches a maximum
level (asymptote). In this context, the Logistic curve helps to
understand important developmental milestones, critical periods for
intervention, and long-term achievement potential.

\subsection{Functional Form of the Logistic
Curve}\label{functional-form-of-the-logistic-curve}

\begin{itemize}
\tightlist
\item
  Equation: ( y(t) = \frac{\alpha}{1 + \exp(-\beta (t - \gamma))} )
\item
  Parameters:

  \begin{itemize}
  \tightlist
  \item
    ( \alpha ): Asymptote (Maximum achievement level)
  \item
    ( \beta ): Growth rate
  \item
    ( \gamma ): Inflection point (Timing of most rapid development)
  \end{itemize}
\end{itemize}

\subsection{Interpretability and Theoretical
Implications}\label{interpretability-and-theoretical-implications}

The Logistic curve provides interpretable parameters directly linked to
theoretical constructs. For example: - The \textbf{inflection point}
indicates the timing of the most rapid change, which can be used to
identify critical developmental periods or windows of opportunity for
educational interventions. - The \textbf{asymptote} represents the
maximum potential achievement level, useful for examining educational
disparities by demographic groups.

\section{Part 2: Logistic Growth Model in R using
brms}\label{part-2-logistic-growth-model-in-r-using-brms}

\subsection{Data Preparation}\label{data-preparation}

To model the logistic growth curve using the brms package in R, we first
need to prepare the data. This example uses the 2011 ECLS-K dataset to
examine academic achievement trajectories. We focus on demographic
differences by SES, race, and gender.

\subsection{Loading Required Packages}\label{loading-required-packages}

\begin{Shaded}
\begin{Highlighting}[]
\CommentTok{\# Install and load necessary packages}
\FunctionTok{install.packages}\NormalTok{(}\FunctionTok{c}\NormalTok{(}\StringTok{"tidyverse"}\NormalTok{, }\StringTok{"brms"}\NormalTok{, }\StringTok{"cmdstanr"}\NormalTok{, }\StringTok{"mice"}\NormalTok{))}
\FunctionTok{library}\NormalTok{(tidyverse)}
\FunctionTok{library}\NormalTok{(brms)}
\FunctionTok{library}\NormalTok{(cmdstanr)  }\CommentTok{\# Required backend for brms}
\FunctionTok{library}\NormalTok{(mice)  }\CommentTok{\# For multiple imputation of missing data}
\end{Highlighting}
\end{Shaded}

\subsection{Data Preparation and
Exploration}\label{data-preparation-and-exploration}

\begin{Shaded}
\begin{Highlighting}[]
\CommentTok{\# Load the dataset}
\CommentTok{\# Assume the dataset is saved as \textquotesingle{}eclsk\_data.csv\textquotesingle{}}
\NormalTok{data }\OtherTok{\textless{}{-}} \FunctionTok{read\_csv}\NormalTok{(}\StringTok{\textquotesingle{}eclsk\_data.csv\textquotesingle{}}\NormalTok{)}

\CommentTok{\# Explore the data}
\FunctionTok{str}\NormalTok{(data)}
\FunctionTok{summary}\NormalTok{(data)}

\CommentTok{\# Subset relevant variables}
\NormalTok{selected\_vars }\OtherTok{\textless{}{-}} \FunctionTok{c}\NormalTok{(}\StringTok{"childid"}\NormalTok{, }\StringTok{"race"}\NormalTok{, }\StringTok{"sex"}\NormalTok{, }\StringTok{"ses"}\NormalTok{, }\StringTok{"age\_entry"}\NormalTok{, }\StringTok{"age\_test"}\NormalTok{, }\StringTok{"math\_score"}\NormalTok{)}
\NormalTok{data\_subset }\OtherTok{\textless{}{-}}\NormalTok{ data[, selected\_vars]}

\CommentTok{\# Handle missing data using multiple imputation}
\NormalTok{imp\_data }\OtherTok{\textless{}{-}} \FunctionTok{mice}\NormalTok{(data\_subset, }\AttributeTok{m =} \DecValTok{5}\NormalTok{, }\AttributeTok{method =} \StringTok{\textquotesingle{}pmm\textquotesingle{}}\NormalTok{, }\AttributeTok{maxit =} \DecValTok{50}\NormalTok{, }\AttributeTok{seed =} \DecValTok{123}\NormalTok{)}
\end{Highlighting}
\end{Shaded}

\subsection{Specifying the Logistic Growth
Model}\label{specifying-the-logistic-growth-model}

The logistic growth model is specified using a custom non-linear
formula:

\begin{Shaded}
\begin{Highlighting}[]
\NormalTok{logistic\_formula }\OtherTok{\textless{}{-}} \FunctionTok{bf}\NormalTok{(}
\NormalTok{  math\_score }\SpecialCharTok{\textasciitilde{}}\NormalTok{ alpha }\SpecialCharTok{/}\NormalTok{ (}\DecValTok{1} \SpecialCharTok{+} \FunctionTok{exp}\NormalTok{(}\SpecialCharTok{{-}}\NormalTok{beta }\SpecialCharTok{*}\NormalTok{ (age\_test }\SpecialCharTok{{-}}\NormalTok{ gamma))),}
\NormalTok{  alpha }\SpecialCharTok{\textasciitilde{}} \DecValTok{1} \SpecialCharTok{+}\NormalTok{ race }\SpecialCharTok{+}\NormalTok{ sex }\SpecialCharTok{+}\NormalTok{ ses }\SpecialCharTok{+}\NormalTok{ age\_entry,}
\NormalTok{  beta }\SpecialCharTok{\textasciitilde{}} \DecValTok{1} \SpecialCharTok{+}\NormalTok{ race }\SpecialCharTok{+}\NormalTok{ sex }\SpecialCharTok{+}\NormalTok{ ses,}
\NormalTok{  gamma }\SpecialCharTok{\textasciitilde{}} \DecValTok{1} \SpecialCharTok{+}\NormalTok{ race }\SpecialCharTok{+}\NormalTok{ sex }\SpecialCharTok{+}\NormalTok{ ses,}
  \AttributeTok{nl =} \ConstantTok{TRUE}
\NormalTok{)}

\CommentTok{\# Priors for interpretability}
\NormalTok{logistic\_priors }\OtherTok{\textless{}{-}} \FunctionTok{c}\NormalTok{(}
  \FunctionTok{prior}\NormalTok{(}\FunctionTok{normal}\NormalTok{(}\DecValTok{0}\NormalTok{, }\DecValTok{10}\NormalTok{), }\AttributeTok{nlpar =} \StringTok{"alpha"}\NormalTok{),}
  \FunctionTok{prior}\NormalTok{(}\FunctionTok{normal}\NormalTok{(}\DecValTok{0}\NormalTok{, }\DecValTok{1}\NormalTok{), }\AttributeTok{nlpar =} \StringTok{"beta"}\NormalTok{),}
  \FunctionTok{prior}\NormalTok{(}\FunctionTok{normal}\NormalTok{(}\DecValTok{0}\NormalTok{, }\DecValTok{10}\NormalTok{), }\AttributeTok{nlpar =} \StringTok{"gamma"}\NormalTok{)}
\NormalTok{)}
\end{Highlighting}
\end{Shaded}

\subsection{Fitting the Model}\label{fitting-the-model}

\begin{Shaded}
\begin{Highlighting}[]
\CommentTok{\# Fit the model using brms}
\NormalTok{fit\_logistic }\OtherTok{\textless{}{-}} \FunctionTok{brm}\NormalTok{(}
  \AttributeTok{formula =}\NormalTok{ logistic\_formula,}
  \AttributeTok{data =} \FunctionTok{complete}\NormalTok{(imp\_data, }\AttributeTok{action =} \DecValTok{1}\NormalTok{),  }\CommentTok{\# Use first imputed dataset}
  \AttributeTok{family =} \FunctionTok{gaussian}\NormalTok{(),}
  \AttributeTok{prior =}\NormalTok{ logistic\_priors,}
  \AttributeTok{chains =} \DecValTok{4}\NormalTok{,}
  \AttributeTok{iter =} \DecValTok{4000}\NormalTok{,}
  \AttributeTok{cores =} \DecValTok{4}\NormalTok{,}
  \AttributeTok{seed =} \DecValTok{123}
\NormalTok{)}
\end{Highlighting}
\end{Shaded}

\subsection{Model Diagnostics and Posterior
Checks}\label{model-diagnostics-and-posterior-checks}

\begin{Shaded}
\begin{Highlighting}[]
\CommentTok{\# Model summary}
\FunctionTok{summary}\NormalTok{(fit\_logistic)}

\CommentTok{\# Trace plots for convergence diagnostics}
\FunctionTok{plot}\NormalTok{(fit\_logistic)}

\CommentTok{\# Posterior predictive checks}
\FunctionTok{pp\_check}\NormalTok{(fit\_logistic)}
\end{Highlighting}
\end{Shaded}

\subsection{Interpretation of
Parameters}\label{interpretation-of-parameters}

\begin{itemize}
\tightlist
\item
  \textbf{Alpha (Asymptote)}: Maximum achievement level.
\item
  \textbf{Beta (Growth Rate)}: Speed of development.
\item
  \textbf{Gamma (Inflection Point)}: Timing of most rapid growth.
\end{itemize}

\subsection{Next Steps}\label{next-steps-1}

The next section will cover implementation of the Logistic Growth Model
in Python using PyMC3. Stay tuned! The next section will cover detailed
coding examples for implementing the Logistic Growth Model in R using
the brms package. Stay tuned!

\bookmarksetup{startatroot}

\chapter*{References}\label{references}
\addcontentsline{toc}{chapter}{References}

\markboth{References}{References}

\phantomsection\label{refs}
\begin{CSLReferences}{0}{1}
\end{CSLReferences}




\end{document}
